\documentclass[addpoints]{exam}

\usepackage{amsmath}
\usepackage{framed}
\usepackage{listings}
\usepackage[margin=1.0in]{geometry}
\usepackage{xcolor}
\definecolor{darkgreen}{RGB}{27, 117, 22}
\usepackage{fontspec}
\header{CS XXXX}{Practice Exam 1}{NetID: \_\_\_\_\_\_\_\_}
\footer{Prof. Ben Rosenberg}{\thepage}{\today}

\lstdefinestyle{mystyle}{  
    commentstyle=\color{gray},
    keywordstyle=\bfseries\color{red},
    numberstyle=\tiny\color{gray},
    stringstyle=\color{darkgreen},
    basicstyle=\ttfamily\footnotesize,
    breakatwhitespace=false,         
    breaklines=true,                 
    captionpos=b,                    
    keepspaces=true,                 
    numbers=left,                    
    numbersep=5pt,                  
    showspaces=false,                
    showstringspaces=false,
    showtabs=false,                  
    tabsize=2,
    frame=single
}

\lstset{language=Python, style=mystyle}

\begin{document}

\begin{center}
\fbox{\fbox{\parbox{5.5in}{\centering
Answer the questions in the spaces provided. If you run out of room
for an answer, continue on the back of the page.}}}
\end{center}

\begin{center}
    \fbox{\gradetable[v][questions]}
\end{center}
    
\begin{center}
\vspace{5mm}
\makebox[0.60\textwidth]{Name:\enspace\hrulefill}

\vspace{5mm}
\makebox[0.60\textwidth]{Section:\enspace\hrulefill}
\end{center}

\newpage

\begin{questions}
\question[8] Name 4 different primitive types.

\vspace{\stretch{1}}

\question \textbf{For loops and while loops}

\begin{parts}

\part[3] When should you use a while loop, and when should you use a for loop?

\vspace{\stretch{1}}

\part[3] In Python, you can always replace a single while loop with a single for loop, as they are functionally equivalent if used correctly. \quad \textbf{T \quad F}

\vspace{\stretch{0.5}}

\end{parts}



\question Given the following piece of code:

\begin{lstlisting}
x1 = [ ... ]
x2 = [ ... ]

output = []
for item1 in x1:
    i = str(item1)
    for item2 in x2:
        result = i + str(item2)
        if not result in output:
            output.append(result)
print(output)
\end{lstlisting}

\begin{parts}

\part[7] What is printed when \texttt{x1 = [1,2,3]} and \texttt{x2 = [2,3,4]}?

\vspace{\stretch{1.25}}

\part[3] Describe, in no more than 3 sentences, what this code does.

\vspace{\stretch{0.75}}

\end{parts}

\newpage

\question[10] Write code that returns the minimum and maximum of a list of integers \verb+S+ with only one pass (only one loop should be used). You are \emph{not} allowed to use the functions \texttt{min} or \texttt{max} in this code (but may do so on other questions). The first line of code is given to you.

\begin{verbatim}
S = [ ... ]     # the contents of S should be irrelevant
\end{verbatim}

\vspace{\stretch{1}}

\question[15] The Fibonacci numbers $F_n$ are defined in the following recursive way: $$F_n = \begin{cases} F_{n-1} + F_{n-2} & n \geq 2 \\ 1 & n = 1 \\ 0 & n = 0 \end{cases}$$

For example, the first 15 Fibonacci numbers are $0, 1, 1, 2, 3, 5, 8, 13, 21, 34, 55, 89, 144, 233, 377$.

Write a piece of code that uses iteration to calculate the $73^\text{rd}$ Fibonacci number and prints it out.

\vspace{\stretch{1}}

\newpage

\question[20] \textbf{Selection sort} is a sorting algorithm that is not very efficient. For an array (i.e., a list) of $n$ elements, it takes time proportional to $n^2$ in order for selection sort to sort it correctly.

The way selection sort works is by repeatedly finding the minimum element of a subset of the array and swapping it with the first element in that subset until the array is sorted. For example, consider the following array: $$[2, 4, 3, 1, 6, 5]$$

On the first iteration, selection sort looks at the whole array and finds that 1 is the minimum element, so it swaps 2 and 1, giving us: $$[1, 4, 3, 2, 6, 5]$$

Then, since selection sort knows that the first element is sorted, it restricts itself to the subset containing the elements $[4, 3, 2, 6, 5]$. In this subset, the minimum element is 2, so it switches 4 and 2: $$[2, 3, 4, 6, 5]$$ 

It continues without switching any more elements for the next two iterations (as 3 and 4 are in their correct places) until it gets to the last two elements, which it swaps: $$[5,6]$$

Expanding the view again gives us the sorted list: $$[1,2,3,4,5,6]$$

Write code that performs selection sort on a list of integers \verb+S+. The first line is written for you.

\begin{verbatim}
S = [ ... ]     # the contents of S should be irrelevant
\end{verbatim}

\vspace{\stretch{1}}

\end{questions}

\end{document}